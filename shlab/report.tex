%!TEX encoding = utf-8
\documentclass[12pt]{report}
\usepackage{geometry}
\usepackage{xcolor}
\usepackage{listings}

\geometry{
	top = 25mm,
	left = 20mm,
	right = 20mm,
	bottom = 20mm
}
\geometry{a4paper}

\pagenumbering{gobble}
\renewcommand{\baselinestretch}{1.3}

\definecolor{mGreen}{rgb}{0,0.6,0}
\definecolor{mGray}{rgb}{0.5,0.5,0.5}
\definecolor{mPurple}{rgb}{0.58,0,0.82}
\definecolor{backgroundColour}{rgb}{0.95,0.95,0.92}

\lstdefinestyle{CStyle}{
	commentstyle=\color{mGreen},
	keywordstyle=\color{mPurple},
	numberstyle=\tiny\color{mGray},
	stringstyle=\color{mGreen},
	basicstyle=\ttfamily\footnotesize,
	breakatwhitespace=false,         
	breaklines=true,                 
	captionpos=b,                    
	keepspaces=true,                 
	numbers=left,                    
	numbersep=-40pt,                  
	showspaces=false,                
	showstringspaces=false,
	showtabs=false,                  
	tabsize=4,
	language=C,
}

\begin{document}
\begin{center}
\textbf{\Large System Programming - Shell Lab Report}\\
\large 2019 Spring, 2017-18570 Sungchan Yi
\end{center}

The first thing that came to my mind was to implement the signal handlers for SIGTSTP and SIGINT, which would be used to exit out of a program inside the shell. The implementation wasn't very hard, I just fetched the PID of the foreground process and sent a kill signal to the process group.

The pdf said that it is required to check the return value of every single system call, so I used a wrapper function that would check the return value for me. I wrote wrapper functions for \texttt{kill}, \texttt{fork}, with the first letter of each system call capitalized. (\texttt{Kill}, \texttt{Fork})

Now the next thing to do was to implement the \texttt{eval}  function. I referred to the code in the textbook, which helped me a lot. Then I figured I had to implement \texttt{builtin\_cmd} function.

The \texttt{builtin\_cmd} function should be able to handle \texttt{quit}, \texttt{jobs}, \texttt{bg} and \texttt{fg}. I just used \texttt{strcmp} to compare the input with the built in commands. And when it is a built in command, execute the command.

If it's not a built in command, call \texttt{Fork} and \texttt{execve}. Furthermore, check whether the process should run in the background and use \texttt{addjob} to add the job to joblist.

Next, it would be natural to implement \texttt{do\_bgfg} for \texttt{bg}, \texttt{fg} commands. This was the hard part. First I would check it the arguments are correct. Then I would determine whether the job id or process id was given. Then depending on the command, change the state of the job accordingly and send a SIGCONT signal. While implementing the part for \texttt{fg} command, I figured that I need the \texttt{waitfg} function because after the \texttt{fg} command, I have to wait for the foreground job to finish. So the shell must wait while the \texttt{fgpid(jobs)} is equal to given \texttt{pid}.

Up to this part, pretty much of the functions were implemented. \texttt{sigchld\_handler} was left. The handler was quite confusing to implement, so I referred to the textbook for more details. The textbook had info on WIFEXITED, WIFSIGNALED, WTERMSIG, WIFSTOPPED, WSTOPSIG function. So I used them to check the state of the child process.

While implementing the \texttt{eval} function, I also found out that I should check the return values of \texttt{sigprocmask} functions, to appropriately block the signals while forking a child process. So I also implemented wrappers for those mask functions, \texttt{Sigemptyset}, \texttt{Sigaddset}, \texttt{Sigprocmask}, and \texttt{Setpgid}.

With those functions implemented, I blocked signals before \texttt{Fork} and unblocked signals before \texttt{execve}, in the child process, unblocked signals after \texttt{addjob} was called.

The lab was pretty much done and I created a python script to check the test cases, and everything was OK.

Then I noticed that I used \texttt{printf} functions, which are not aync-signal-safe. So in the signal handlers, I tried to use the \texttt{write} function, the only async-signal-safe function for printing. But to use a format string in write, I had to write to a buffer and then use \texttt{write}. But to write to a buffer, I needed \texttt{sprintf} function, which is not async-signal-safe. So I thought of using \texttt{for} loop to assign each character to a buffer, but that would violate the rule: "Keep your handlers simple as possible". So I disassembled the \texttt{tshref} and checked that they used \texttt{printf}. So I also stuck to \texttt{printf}.

This homework was very interesting because I am really fond of the shell in Linux. It was a great chance writing my own shell.

I asked TAs for how to handle the async-signal-safe functions, TAs suggested that I use the safe I/O functions. When I searched for SIO in the textbook, it used \texttt{\#include "csapp.h"}. So I went online and downloaded \texttt{csapp.h}, \texttt{csapp.c} and used the SIO functions from the header file. The \texttt{sigchld\_handler} function is now async-signal-safe.
\end{document}