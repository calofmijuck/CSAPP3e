%!TEX encoding = utf-8
\documentclass[12pt]{report}
\usepackage{geometry}
\usepackage{xcolor}
\usepackage{listings}

\geometry{
	top = 25mm,
	left = 20mm,
	right = 20mm,
	bottom = 20mm
}
\geometry{a4paper}

\pagenumbering{gobble}
\renewcommand{\baselinestretch}{1.3}

\definecolor{mGreen}{rgb}{0,0.6,0}
\definecolor{mGray}{rgb}{0.5,0.5,0.5}
\definecolor{mPurple}{rgb}{0.58,0,0.82}
\definecolor{backgroundColour}{rgb}{0.95,0.95,0.92}

\lstdefinestyle{CStyle}{
	commentstyle=\color{mGreen},
	keywordstyle=\color{mPurple},
	numberstyle=\tiny\color{mGray},
	stringstyle=\color{mGreen},
	basicstyle=\ttfamily\footnotesize,
	breakatwhitespace=false,         
	breaklines=true,                 
	captionpos=b,                    
	keepspaces=true,                 
	numbers=left,                    
	numbersep=10pt,                  
	showspaces=false,                
	showstringspaces=false,
	showtabs=false,                  
	tabsize=4,
	language=C,
}

\begin{document}
\begin{center}
\textbf{\Large System Programming - Malloc Lab Report}\\
\large 2019 Spring, 2017-18570 Sungchan Yi
\end{center}
After looking at the code from the book, I decided to implement a segregated list. I had to define a bunch of macros, I mostly got the names from the book. Most of the explanation is in the comments.
\begin{lstlisting}[style=Cstyle]
#define ALIGNMENT 8
#define ALIGN(size) (((size) + (ALIGNMENT-1)) & ~0x7)

// Macros from code in textbook + My Macros
#define SIZEW 4
#define SIZEQ 8
#define INITCHUNKSIZE (1 << 6)
#define CHUNKSIZE (1 << 12)
#define LIST_N 20

// Pack size and alloc bit into a word
#define PACK(size, alloc) ((size) | (alloc))

// Read from ptr
#define GET(ptr) (*(unsigned int *)(ptr))
// write val to p
#define PUT(p, val) (*(unsigned int *)(p) = (val))

// size, alloc bit, tag bit at ptr
#define GET_SIZE(ptr) (GET(ptr) & ~0x7)
#define GET_ALLOC(ptr) (GET(ptr) & 1)

// Address of block header / footer
#define HDRP(ptr) ((char *)(ptr) - SIZEW)
#define FTRP(ptr) ((char *)(ptr) + GET_SIZE(HDRP(ptr)) - SIZEQ)

// next block, prev block
#define NEXT_BLKP(ptr) ((char *)(ptr) + GET_SIZE((char *)(ptr) - SIZEW))
#define PREV_BLKP(ptr) ((char *)(ptr) - GET_SIZE((char *)(ptr) - SIZEQ))

// Next, prev free block entry
#define PRED_PTR(ptr) ((char *)(ptr))
#define SUCC_PTR(ptr) ((char *)(ptr) + SIZEW)

// Next, prev of free block on seg list
#define PRED(ptr) (*(char **)(ptr))
#define SUCC(ptr) (*(char **)(SUCC_PTR(ptr)))

// Set Pointer
#define SET_PTR(p, ptr) (*(unsigned int *)(p) = (unsigned int)(ptr))
\end{lstlisting}~\\
There were so many macros, but because macros are usually a great source of bugs, I tried not to use them very much... I decided to have 20 range of sizes for the segregated list, so I used a double pointer.
\begin{lstlisting}[style=Cstyle]
void *l0 = 0, *l1 = 0, *l2 = 0, *l3 = 0, *l4 = 0, *l5 = 0, *l6 = 0, *l7 = 0, *l8 = 0, *l9 = 0, *l10 = 0, *l11 = 0, *l12 = 0, *l13 = 0, *l14 = 0, *l15 = 0, *l16 = 0, *l17 = 0, *l18 = 0, *l19 = 0;
void **seg_list = &l19; // Now works as a pointer to a pointer array
\end{lstlisting}~\\
\texttt{seg\_list} will point to the list. \texttt{*l}$n$ can be referenced by \texttt{seg\_list[n]}.\\
Initialize the procedure, allocating the heap, setting an initial block etc.
\begin{lstlisting}[style=Cstyle]
int mm_init(void) {
	int i = 0;
	char *heap_st; // track the start of the heap
	
	// Initialize segregated list
	while(i < LIST_N) seg_list[i++] = NULL;
	
	// Allocate memory on heap
	if((long) (heap_st = mem_sbrk(4 * SIZEW)) == -1) return -1;
	
	// Initial block
	PUT(heap_st, 0);
	PUT(heap_st + (1 * SIZEW), PACK(SIZEQ, 1));
	PUT(heap_st + (2 * SIZEW), PACK(SIZEQ, 1));
	PUT(heap_st + (3 * SIZEW), PACK(0, 1));
	
	// extend heap
	if(!extend_heap(INITCHUNKSIZE)) return -1;
	return 0;
}
\end{lstlisting}~\\
And here was the fun part, extending the heap and inserting, deleting elements to the segregated list. For extending the heap, I needed to consider the alignment requirements. After extending the heap, consider it as a free block and insert it into the list and coalesce it.\\
The insertion and deletion procedure was quite similar to the normal insertion/deletion procedures for linked lists. Furthermore, the segregation list range was decided by $2^k$. To decide which list the block should go into, call a subroutine \texttt{select\_list}.
\begin{lstlisting}[style=Cstyle]
static void *extend_heap(size_t size) {
	void *ptr;
	size_t nsize = ALIGN(size); // alignment of new size
	// In case it fails
	if((ptr = mem_sbrk(nsize)) == (void *) -1) return NULL;
	
	PUT(HDRP(ptr), PACK(nsize, 0));
	PUT(FTRP(ptr), PACK(nsize, 0)); // Update header and footer
	PUT(HDRP(NEXT_BLKP(ptr)), PACK(0, 1));
	insert(ptr, nsize); // insert newly extended area to the list
	return coalesce(ptr); // coalesce if necessary
}

// Select from segregated list
static int select_list(int size) {
	int idx = 0;
	while(idx < LIST_N - 1 && size > 1) {
		size >>= 1;
		idx++;
	}
	return idx;
}

// Insert into segregated list
static void insert(void *ptr, size_t size) {
	void *search_p = ptr;
	void *insert_p = NULL;
	int idx = select_list(size); // Select from segregated list
	
	// Search for a place to fit
	search_p = seg_list[idx];
	while(search_p && (size > GET_SIZE(HDRP(search_p)))) {
		insert_p = search_p;
		search_p = PRED(search_p);
	}
	
	// Set next, prev
	if(search_p) {
		SET_PTR(PRED_PTR(ptr), search_p);
		SET_PTR(SUCC_PTR(search_p), ptr);
		if(insert_p) {
			SET_PTR(SUCC_PTR(ptr), insert_p);
			SET_PTR(PRED_PTR(insert_p), ptr);
		} else {
			SET_PTR(SUCC_PTR(ptr), NULL);
			seg_list[idx] = ptr;
		}
	} else {
		SET_PTR(PRED_PTR(ptr), NULL);
		if(insert_p) {
			SET_PTR(SUCC_PTR(ptr), insert_p);
			SET_PTR(PRED_PTR(insert_p), ptr);
		} else {
			SET_PTR(SUCC_PTR(ptr), NULL);
			seg_list[idx] = ptr;
		}
	}
	
	return;
}

// Delete from segregated list
static void delete(void *ptr) {
	size_t size = GET_SIZE(HDRP(ptr));
	int idx = select_list(size); // Select from seg list
	
	// Deletion works just like linked lists
	if(PRED(ptr)) {
		if(SUCC(ptr)) {
			// set succ of pred to the succ of current
			SET_PTR(SUCC_PTR(PRED(ptr)), SUCC(ptr));
			// set pred of succ to the pred of current
			SET_PTR(PRED_PTR(SUCC(ptr)), PRED(ptr));
		} else {
			SET_PTR(SUCC_PTR(PRED(ptr)), NULL);
			seg_list[idx] = PRED(ptr);
		}
	} else {
		if(SUCC(ptr)) SET_PTR(PRED_PTR(SUCC(ptr)), NULL);
		else seg_list[idx] = NULL;
	}
	return;
}
\end{lstlisting}~\\
For coalescing, identify all 4 cases and handle them separately.
\begin{lstlisting}[style=Cstyle]
static void *coalesce(void *ptr) {
	// allocation status of prev, next
	size_t prev = GET_ALLOC(HDRP(PREV_BLKP(ptr)));
	size_t next = GET_ALLOC(HDRP(NEXT_BLKP(ptr)));
	size_t size = GET_SIZE(HDRP(ptr));
	
	if(prev && next) return ptr; // Case 1
	delete(ptr);
	if(prev && !next) { // Case 2
		delete(NEXT_BLKP(ptr)); // delete next block
		size += GET_SIZE(HDRP(NEXT_BLKP(ptr))); // add size
		PUT(HDRP(ptr), PACK(size, 0)); // update header
		PUT(FTRP(ptr), PACK(size, 0));
	} else if(!prev && next) { // Case 3
		delete(PREV_BLKP(ptr)); // delete prev block
		size += GET_SIZE(HDRP(PREV_BLKP(ptr))); // add size
		PUT(FTRP(ptr), PACK(size, 0)); // update footer
		PUT(HDRP(PREV_BLKP(ptr)), PACK(size, 0)); // update header
		ptr = PREV_BLKP(ptr); // set to prev block
	} else { // Case 4
		delete(PREV_BLKP(ptr)); // delete prev block
		delete(NEXT_BLKP(ptr)); // delete next block
		size += GET_SIZE(HDRP(NEXT_BLKP(ptr)));
		size += GET_SIZE(HDRP(PREV_BLKP(ptr))); // add size
		PUT(HDRP(PREV_BLKP(ptr)), PACK(size, 0));
		PUT(FTRP(NEXT_BLKP(ptr)), PACK(size, 0));
		ptr = PREV_BLKP(ptr); // set to prev block
	}
	insert(ptr, size); // insert newly coalesced block
	return ptr;
}
\end{lstlisting}~\\
And the important part was actually implementing malloc. To find a free block in the segregated list, call a subroutine \texttt{search\_block}.
\begin{lstlisting}[style=Cstyle]
static void *search_block(void* ptr, int asize) {
	size_t ssize = asize; // search size
	int idx = 0;
	for(; idx < LIST_N; ++idx, ssize >>= 1) {
		if(idx == LIST_N - 1 || ((ssize <= 1) && seg_list[idx])) {
			ptr = seg_list[idx];
			// ignore small blocks
            while(ptr && asize > GET_SIZE(HDRP(ptr))) {
				ptr = PRED(ptr);
			}
			if(ptr) break;
		}
	}
	return ptr;
}

void *mm_malloc(size_t size) {
	size_t asize; // adjust size
	size_t ext_size; // extended size
	void *ptr = NULL;
	
	// size 0
	if(!size) return NULL;
	if(size <= SIZEQ) asize = 2 * SIZEQ;
	else asize = ALIGN(size + SIZEQ);
	
	ptr = search_block(ptr, asize); // Search for free block in seglist
	
	// if not found, extend heap
	if(!ptr) {
		ext_size = asize > CHUNKSIZE ? asize : CHUNKSIZE;
		if(!(ptr = extend_heap(ext_size))) return NULL;
	}
	
	// Place block, split if necessary
	// splitting will be taken care of by the place function
	ptr = place(ptr, asize);
	return ptr;
}
\end{lstlisting}~\\
Free was quite simple at first thought, but I had a hard time debugging because I forgot to put \texttt{HDRP(ptr)} in the second line...
\begin{lstlisting}[style=Cstyle]
void mm_free(void *ptr) {
	size_t size = GET_SIZE(HDRP(ptr)); // HDRP(ptr) !!!
	PUT(HDRP(ptr), PACK(size, 0));
	PUT(FTRP(ptr), PACK(size, 0)); // Update header / footer
	insert(ptr, size); // insert to seg_list
	coalesce(ptr); // coalesce free blocks
	return;
}
\end{lstlisting}~\\
Now for the reallocation, use the implemented \texttt{mm\_malloc} and \texttt{mm\_free}. Simply follow the C standards.
\begin{lstlisting}[style=Cstyle]
void *mm_realloc(void *ptr, size_t size) {
	size_t oldsize;
	void *newptr;
	
	if(!size) { // realloc(ptr, 0) is equal to free
		mm_free(ptr);
		return 0;
	}
	size += 1 << 7; // add size for future reallocation
	if(!ptr) return mm_malloc(size); // realloc(NULL, size) is malloc(size)
	newptr = mm_malloc(size);
	if(!newptr) return 0; // if fail, return 0
	
	oldsize = size < GET_SIZE(HDRP(ptr)) ? size : GET_SIZE(HDRP(ptr));
	memcpy(newptr, ptr, oldsize); // copy old data
	mm_free(ptr); // free the old block
	return newptr;
}
\end{lstlisting}~\\
This lab was particularly hard due to so many macros that I had to use. I even had to waste hours due to a mistake in the macro. But after writing the code, and when I reviewed it, it seems to me that macros are really necessary. The code looks much cleaner thanks to the macro. I should really be careful about them. Moreover, simulating malloc was a great experience, manipulating data at the kernel level.\footnote{I'm not sure if I'm allowed to say this...} I kept having the urge to use malloc whenever I tried to create a new node, but since it's already free memory, it didn't matter!
\end{document}